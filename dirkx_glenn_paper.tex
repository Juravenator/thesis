\documentclass[journal,compsoc,a4paper]{IEEEtran}
\pagenumbering{gobble}
\usepackage{graphicx}
\graphicspath{{images/}}
\usepackage[ampersand]{easylist}
\usepackage{scrextend}

% correct bad hyphenation here
\hyphenation{AjaXell}

\begin{document}
\title{Upgrading the Interface and Developer Tools of the Trigger Supervisor Software Framework of the CMS experiment at CERN}
\author{Glenn~Dirkx,\IEEEmembership{~Student,~KU Leuven,~glenn.dirkx@student.kuleuven.be}\\
        Dr.~Christos~Lazaridis,\IEEEmembership{~University~of~Wisconsin,~Christos.Lazaridis@cern.ch}\\
        Dr.~Peter~Karsmakers,\IEEEmembership{~KU Leuven, ~peter.karsmakers@kuleuven.be }% <-this % stops a space
        }

\IEEEcompsoctitleabstractindextext{%
\begin{abstract}
%\boldmath
% Het abstract is het eerste deel van de paper en geeft een korte samenvatting
% van de paper in circa 150 woorden.
% Het abstract is een zeer belangrijk gedeelte van de paper aangezien de meeste
% wetenschappers op basis van het lezen van het abstract oordelen of de paper al
% dan niet interessant is om verder te lezen. Het abstract bevat zowel een korte
% probleemstelling, de gevolgde aanpak als de bekomen resultaten.

The Compact Muon Solenoid (CMS) Trigger Supervisor (TS) is a software framework that has been designed to
handle the CMS Level-1 trigger setup, configuration and monitoring during data
taking as well as all communications with the main run control of CMS.

The interface consists of a web-based GUI rendered by a back-end C++ framework
(AjaXell) and a front-end JavaScript framework (Dojo).
These provide developers with the tools they need to to write their own custom
control panels.

However, currently there is much frustration with this framework given the age
of the Dojo library and the various hacks needed to implement modern use cases.

The task at hand is to renew this library and its developer tools, updating it
to use the newest standards and technologies, while maintaining full
compatibility with legacy code.\\

This paper describes the requirements, development process, and changes to
this framework that were included in the upgrade from v2.x to v3.x.
\end{abstract}

\begin{IEEEkeywords}
CERN, CMS, L1 Trigger, C++, Polymer, Web Components.
\end{IEEEkeywords}}

\maketitle

\IEEEdisplaynotcompsoctitleabstractindextext

\IEEEpeerreviewmaketitle

\section{Introduction}

% Some journals put the first two words in caps:
% \IEEEPARstart{T}{his demo} file is ....
%
% In deze sectie wordt het artikel ingeleid en wordt het probleem
% toegelicht waarvoor het werk in dit artikel bedoeld is. Desgewenst kan hierin
% ook een ruimere domein-specifieke inleiding opgenomen worden. Belangrijk is om
% de lezer in deze sectie te overtuigen van het probleem en de noodzaak van uw werk.
% Eventueel kunnen reeds verwijzingen naar relevant ander werk opgenomen worden.
The CMS experiment at the European Organization for Nuclear Research (CERN)
consists of many components. One of them is the Level-1 (L1) trigger, designed to filter
the enormous amount of data generated by the proton-proton collisions at the
experiment (currently around 80TB/s\cite{CMS_Experiment2}).
The Trigger Supervisor (TS) is a software project that aims to control the L1 trigger.
This includes setup, configuration, and monitoring before and during during data
taking. It allows for controlling various aspects of the L1 trigger using panels
in a web interface. This interface is custom built for each use case, although
some generic panels exist (commands, operations, \ldots).

The main software library that facilitates this, is AjaXell. It provides the
developer tools to make a custom panel. It does, however, have a few problems.

Firstly, the Dojo library that AjaXell uses to render the panels is old. It
misses functionality required for modern use cases and it even starts to break
down on modern browsers. For example, a modal dialog does not render in Google
Chrome 44 but the transparent background that usually forces the user to make
a choice in the modal does render. This effectively blocks the user from doing
anything and forces the user to reload the page and start over.
Today, when a developer wants to provide new functionality, the solution is to
just manually write the HTML and JavaScript.

Secondly, the current state of affairs requires developers to write everything
in one big C++ file. Panels consist of many languages (C++, HTML, JavaScript,
and CSS) and combined with the fact that much functionality is written manually,
this results in very messy and unreadable code.
Several existing panels are very difficult to modify because of this.

This paper describes the changes that have been made to the TS to solve these
problems.

\section{Related Work}
The full original design of the Trigger Supervisor framework can be consulted in
the PHD thesis of Ildefons Magrans de Abril\cite{TS_PHD}. This thesis describes
both the hardware and software design decisions that were made and provide more
detail about the TS in general, as this paper merely describes the operator
interface redesign.

It's also recommended to read the Phase II upgrade technical proposal of the
CMS experiment\cite{TS_Phase2} as it describes the upgrade from Phase I
\cite{CMS_Experiment} and its new architecture and ideas that the work described
here accompanies.

% \section{Inhoudelijke secties}
% Na introductie en related work komt de bespreking van het eigen werk.
% De structuur van deze secties hangt uiteraard sterk af van de aard van jouw onderzoek.
% Volgende secties kunnen in deze of een iets andere volgorde aan bod komen.
% Zorg hier vooral voor een duidelijke structuur die door een lezer gemakkelijk gevolgd kan worden:
% o Specifieke aanpak (eventuele requirements, ...).
% o Theoretische afleidingen.
% o Technische opstelling.
% o Simulaties.
% o Experiment/ Evaluaties.
% o Resultaten (geen bespreking; enkel de droge resultaten).
% o Discussion/Bespreking: Het laatste deel van deze secties bespreekt,
% bediscussieert of concludeert vaak de bekomen resultaten. Dit is niet te
% verwarren met de volgende sectie ‘conclusion’.

\section{Functional Requirements}
Ideally the result would be a new framework, much more powerful than its
predecessor, that yet manages to achieve 100\% compatibility with legacy code.

The main objectives are cleaner code, better maintainability, better documentation,
and easier development. Making the framework easier to develop on, will invite
developers to write more advanced code.

\section{Upgrade options}
Although in the final stage there will be no more legacy code, old code
must still remain functional in the new environment to allow for a smooth
transition.

This limits the available options at the back-end. Because of this, only extra
code to the existing C++ codebase can be added. Changes are not possible.

On the front-end side there are a bit more possibilities.
The only important requirement is that whatever the new code looks like, the old
Dojo code must be able to run alongside it.
\subsubsection{Dojo v1.10}
It is not possible to just upgrade to a new version of Dojo. Currently AjaXell
uses Dojo 0.4 and starting from Dojo 0.9 there has been a major API change.
Two different versions of Dojo cannot run concurrently since they still share
a lot of function calls.

Also this approach would not solve any of the currently existing problems.
Interfaces would still have messy code and frustrated developers.
% \subsubsection{AngularJS}
\subsubsection{Web Components}
Web Components\cite{WebComponentsW3C}\cite{WebcomponentsMozilla} are additions to the HTML5 standard. They
enable a developer to develop custom HTML tags, the idea is to mitigate the
`div soup` problem\cite{DivSoup} where the web application's source code increases
exponentially in size as the complexity of the app increases.

This standardizes an approach seen in many modern JavaScript frameworks such as
AngularJS, Ember.js, Knockout.js, Dojo, and Backbone.js. These all allow a developer to
declare specialized `elements` in order to make developing a smart web application
easier.
However, by relying on the Web Components standard it can be safely assumed the
problem encountered with Dojo 0.9, which introduced breaking API changes, will
not occur again.
Despite being a new standard, support for all CERN-supported
browsers (firefox ESR 24-current) can be achieved using the webcomponents.js polyfill.
\subsubsection{Polymer}
Polymer is a relatively new library, built directly on the Web Components
standards, developed by Google. It represents the way Google thinks Web
Components should be used.
The reason Polymer is very useful is that it has the potential
to allow us to introduce proper Separation of Concerns (SoC) principles
(\ref{Separation of Concerns}) to the development environment.

\section{The new development environment}
\subsection{Separation of Concerns}
\label{Separation of Concerns}
SoC is a design primitive, dictating a modular design of software. This has
been implemented in three ways.

Firstly, different syntaxes now are housed in their own files. This allows for
significantly less messy code and enables us to implement specific optimizations
for each language (for example a CSS pre- and post-processor).

Secondly, the developer is not limited to one source file for each syntax. If
circumstances would make some code easier to manage if it is housed across
multiple files this is now possible. An example of this would be a panel with
multiple specialized sections. Separating these sections will make the code
easier to read and maintain.

Thirdly, this approach pushes developers to separate data from markup. This is
a very good thing as it causes the code to once again be much more readable.
By having the C++ code only produce the necessary data and putting all rendering
and interaction on the front-end a developer can also safely replace rendering logic or
user interaction flow without having to worry about data generation.

\subsection{Build Cycle}
Instead of loading all the separated files individually at runtime, they will
instead be compiled together at compile-time. This will improve loading speeds.
The tool used to do this is Grunt
(http://gruntjs.com/), a task runner built on nodeJS that is used to compile,
minify, lint, unit-test,\ldots front-end code languages.
It has very wide community adoption, which results in a very rich set of tools
available for use.
\subsection{Code optimization}
Now that every code language is housed in specialized files some
optimizations can be done on them at compile-time. The main objective of these optimizations
is to achieve as much browser compatibility as possible.
\subsubsection{JavaScript}
In order to ensure compatibility with all required browsers all JavaScript code
is transpiled by Babel (https://babeljs.io/). This will ensure that newer syntax,
like ECMAScript 2016 (ES7), will be transpiled into a more compatible equivalent.

Also the JavaScript code will be transpiled by UglifyJS
(https://github.com/mishoo/UglifyJS). This will implement various code
optimizations\cite{UglifyJSCompressor} making the code faster.
\subsubsection{CSS}
Developers are given the possibility to write SASS code, an extension of the CSS
syntax, that will be transpiled into CSS on compile-time using libsass
(http://sass-lang.com/libsass).

Also Grunt will automatically add vendor-specific prefixes to CSS properties to
maintain the required browser compatibility using a tool called autoprefixer
(https://github.com/postcss/autoprefixer).

\subsection{Code sharing}
Code duplication should be minimized as much as possible. Code that is used
frequently is therefore moved to a separate code repository available for
anyone to use. These include things like chart libraries, layout frameworks, and
some in-house components such as an auto-updater.

\section{Documentation}
Documentation is something commonly taken too lightly. Fortunately there are some
tools not only to make good documentation but also to encourage developers
later on to write proper documentation.

Most of the documentation is housed along with the source code itself. The goal
is to minimize separation of code and documentation as this easily leads to
inconsistencies between code and documentation.

\subsection{Inline Documentation}
Advantages of inline documentation are the reduced chances for outdated
documentation and being able to enrich source code with typed annotations
\cite{JS_Annotations}.

Source code consists of C++, JavaScript, HTML, and CSS code. The inline
documentation described here is applicable to the last three.

The syntax used to document JavaScript code is called JSDocs and is currently
at version 3\cite{JS_Annotations}\cite{JSDoc}. It provides us with a rich set of expressions
enabling a developer to write documentation comparable to JavaDoc.

In addition there are specific points in the source code where a developer can
provide code examples and extra directives to document HTML and CSS code. This
is however a non-standard method since there is no standardized way to inline-
document any of the other languages.

\subsection{Global level}
The global level is the only level where documentation is separated from the source code.
This houses documentation aimed to teach users and developers the concepts and
ways of thinking regarding this codebase.
It teaches developers the basics of the structure they will be developing in and
the philosophy behind this structure.

This global documentation level is built using Sphinx (http://www.sphinx-doc.org/)
and provides a single point of entry for people looking for documentation and
will guide readers to the next level of documentation when they are ready for it.

\subsection{Package level}
The codebase is composed of a number of packages. Each package automatically
generates documentation describing its content and capabilities.

This documentation is generated in the Grunt build cycle described earlier.
It loads and interprets every component of the package and generates a summary
page giving a general overview and pointing to several useful resources for each
component such as the documentation on the element level, a link to the code
repository, and a link to a live demo of the component if available.

The code it uses to render this documentation is housed in the source code of
each component. It gets interpreted by Grunt and is then compiled in the package
documentation page.

\subsection{Element level}
The lowest level of documentation is documentation of individual web components.
This level is also auto-documented from the component's source code.
But unlike the documentation on the package level, where documentation is
generated on compile time, the documentation here is rendered on the fly.

This is done by using a specialized web component called `iron-component-page`
(https://elements.polymer-project.org/elements/iron-component-page).
It interprets the source code of the component and comments left by the developer
and compiles this into a documentation page.

This documentation provides an overview of all the properties and available
calls of this component. It can also provide code examples and even live
demos.

\section{Results}
\subsection{Loading times}
\label{Loading times}
Table \ref{tbl:loadingtimes} shows an overview of the initial full page loading
times for the legacy TS (version 2.1.0) and the new TS (version 3.4.0). That is,
a page load from a new browser tab with all caches removed.

This test is performed with the timeline panel of Google Chrome 50.0.2661.86 (64-bit).

\begin{table}
  \begin{center}
    \begin{tabular}{| l | l | l | l | l |}
    \hline
     & TS 2.1.0 & TS 3.4.0 & difference & difference (\%) \\ \hline
    \textbf{Loading} & 44.5ms & 112.4ms & +67.9ms & +152.58\%  \\ \hline
    \textbf{Scripting} & 1227.6ms & 1187.6ms & -40ms & -3,26\% \\ \hline
    \textbf{Rendering} & 29.7ms & 171.0ms & +141.3ms & +475,76\% \\ \hline
    \textbf{Painting} & 7.5ms & 36.1ms & +28.6ms & +381.33\% \\ \hline
    \textbf{Other} & 106.4ms & 335.9ms & +229.5ms & +215,69\% \\ \hline
    \textbf{Idle} & 213.6ms & 775.1ms & +561.5ms & +262,87\% \\ \hline \hline
    \textbf{Total} & \textbf{1.63s} & \textbf{2.62s} & \textbf{+990ms} & \textbf{+60,73\%} \\ \hline
    \end{tabular}
  \end{center}
  \caption{Page loading times for TS 2.x and 3.x}
  \label{tbl:loadingtimes}
\end{table}

It is expected that the TS 3.x has higher values for everything in this table,
because it loads two front-end libraries (Dojo \& Polymer).

Notable is the decrease of scripting time for the TS 3.x relative to the
TS 2.x. This is because Dojo is minified and packaged into one JavaScript file
in the TS 3.x release, where as in the TS 2.x release it was not.
Also, because this test is performed in Google Chrome, which has native support
for Web Components, very little scripting needs to be done.
This result will be different in other browsers like Mozilla Firefox, where
Web Components support needs to be emulated. Then again, the lazy loading system
largely removes this overhead from the initial page loading time, so only minor
differences would be expected here.

Rendering time has increased the most going from TS 2.x to TS 3.x. This makes
sense as Polymer renders everything on the front-end, whereas Dojo used to render
everything server-side.
During initial page load this rendering load is primarily caused by the rendering
of the left side menu.
The increase of painting time follows the same logic as the rendering time.

Also notable is the increase of idle time. This means that the browser needs to
wait for a task to finish before it can start another.
This is caused because the TS 3.x loads the default panel after the initial page
load. Which means the TS makes extra network request, to fetch an interface panel,
right after initializing. This is counted with the initial page load.
TS 2.x just shows a blank page, it loads no default panel.
Because the browser needs to wait for the extra network requests to finish before
it can render the default panel, the idle time goes up by a lot.

In total, the initial page loading time increased with about 60\%, which is an
acceptable increase given the new TS runs 2 libraries concurrently.

\subsection{CPU consumption}
Both TS releases have negligible CPU usage when doing a fresh page load, and
stay at 0\% CPU usage when the user is not interacting with the system.

TS 3.x uses hardware acceleration for it's animations since they are all made
using CSS transform properties or using Web Animations\cite{webanimations}.
The only exception to this is the `paper-spinner` element. Which displays a
loading animation.
The TS 2.x release did not have any animations.

\subsection{Memory consumption}
The Dojo library of TS 2.x contained memory leaks, and could lead to a web
browser using an excessive amount of memory when an interface was used for
a long duration of time.

Unfortunately, legacy
panels in the new TS still suffer from this memory leak. This is because the
circular references causing the memory leak reside in the Dojo library itself,
and thus would be impractical to address.
Therefore, any interface that included auto-refreshes had the highest priority
to be converted to a new TS 3.x interface.

Because TS 3.x uses client-side interface rendering rather than server-side as
the TS 2.x did, it uses more memory from the browser.

In TS 3.x the memory used
by an interface panel will be released after there is a switch to another panel.
It is also known that in TS 2.x the memory consumption grows linearly with the
amount of panels loaded by the user.

To test the difference in memory consumption, both TS versions were opened in
a new tab while memory consumption is monitored. No panels are loaded, the
interfaces are just left for 120s. The mean memory consumption in those 120s is
then taken as the mean memory consumption for that TS release.
The results of this test are shown in table \ref{tbl:memoryusage}.

\begin{table}
  \begin{center}
    \begin{tabular}{| l | l | l |}
    \hline
     & Google Chrome & Firefox \\ \hline
    \textbf{TS 2.1.0 (Dojo)} & 20.051MB & 7.06MB \\ \hline
    \textbf{TS 3.4.0 (Dojo + Polymer)} & 24.564MB & 10.96MB \\ \hline
    \textbf{difference} & +4.513MB & +3.9MB \\ \hline
    \textbf{difference (\%)} & \textbf{+22.51\%} & \textbf{+55.24\%} \\ \hline
    \end{tabular}
  \end{center}
  \caption{Memory usage for TS 2.x and 3.x in Mozilla Firefox and Google Chrome}
  \label{tbl:memoryusage}
\end{table}

\section{Functionality}
TS 3.x has functionally more capabilities for the interface than TS 2.x had.
More importantly, the TS interface is now no longer bound to one framework.
Any Web Component can be used, and extra functionality can be developed in-house.
This unlike TS 2.x where developers were functionally bound to the elements the Dojo
developers provided.

This makes TS 3.x far more easy to change, and thus more ready for the future.

\section{SDK improvements}
The fact that multiple programming languages are no longer placed into one file,
but distributed across multiple files, makes the developing an interface panel
a lot easier.

The Web Components approach to build interfaces gives developers a set of
powerful tools that are easy to use and extend.
% \subsection{Decreased development time}

% \section{Documentation}

\section{Developed panels}
The Control Panels are a set of custom interfaces, developed for an individual cell.
The other panels however occur on every cell. And are upgraded as part of the
new TS release.
\subsection{Commands}
The new commands panel use the `command-input` element for its input. Making it
easily extendible to understand more input types (e.g. vectors).
Currently it understands number, int, long, unsigned int, unsigned long, short,
unsigned short, string, double, and float input.

\subsection{Operations}
The TS 2.x operations panel had some problems with auto-updating.
The state diagram tended to update very late, if it updated at all.
Result data and new available commands usually took more than 10 seconds to
show up in the interface.

The new operations panel is now far more responsive.
The state diagram is available when clicking on an icon, as it was deemed a
waste of space to show it by default.

\subsection{Flashlists}

The flashlist panels now have a user-configurable auto-update function.
The flashlist can deploy custom renderers in the table depending on the data type,
for example a date will be shown as relative time (e.g. 9 minutes ago), instead
of just showing a time stamp. This list of custom renderers can be extended
easily.
% \subsection{Demos}


\section{Conclusion}
% Deze sectie vat, als antwoord op het abstract, de resultaten
% en de bijdragen van deze paper samen. Vaak start de conclusie met een zeer
% bondige herhaling van het doel van de paper, en vat daarna de wetenschappelijke
% bijdrage samen. Een lezer zou in principe na het lezen van het abstract en de
% conclusies de essentie van de paper moeten kennen.
The main objective was to upgrade the TS to be able to provide more advanced
interfaces, and to keep compatibility with legacy interfaces.

The new interface engine has achieved 100\% backwards compatibility, while
providing a completely new way to develop new interfaces.

This new interface engine and can be easily extended and is ready for any future
use-cases as it is built to change. The developers are not bound to the
functionality of one framework, rather it is build on open standards and thus
ensures maximum compatibility with future technologies.

The interface developers now have internal, semi auto-generated, documentation
at their disposal and have an active community on the world wide web to fall
back to.




% if have a single appendix:
%\appendix[Proof of the Zonklar Equations]
% or
%\appendix  % for no appendix heading
% do not use \section anymore after \appendix, only \section*
% is possibly needed

% use appendices with more than one appendix
% then use \section to start each appendix
% you must declare a \section before using any
% \subsection or using \label (\appendices by itself
% starts a section numbered zero.)
%


% \appendices
% \section{Proof of the First Zonklar Equation}
% Appendix one text goes here.
%
% % you can choose not to have a title for an appendix
% % if you want by leaving the argument blank
% \section{}
% Appendix two text goes here.


% use section* for acknowledgement
\ifCLASSOPTIONcompsoc
  % The Computer Society usually uses the plural form
  \section*{Acknowledgments}
\else
  % regular IEEE prefers the singular form
  \section*{Acknowledgment}
\fi


% The authors would like to thank...
%
% In sommige gevallen zijn de auteurs bepaalde personen of organisaties dank verschuldigd.
% Bijvoorbeeld in geval van (technische) hulp binnen het bedrijf,
% kunnen deze personen hier genoemd worden.
% Mede-auteurs (zoals in dit geval je promotor of contactdocent bedank je niet)

% I would like to thank the following people for their assistance during this
% project:\\
% \textbf{Evangelos Paradas} for his guidance trough the architecture of the TS
% and pointing me to useful resources.\\
% \textbf{Alessandro Thea} for his advice on how to proceed with implementing new
% functionalities and his supply of motivation and inspiration.
%
% Furthermore I would like to express my thanks to the entire Online Software team
% for the freedom and trust I've been given that allowed this project to get as
% far as it has.
I would like to thank the following people for their assistance during this
project:\\

\begin{labeling}{Evangelos Paradasss}
\item [\textbf{Christos Lazaridis}] for being a great mentor and for not getting mad
when I break the nightlies or even SVN itself.\\
\item [\textbf{Alessandro Thea}] for his advice on how to proceed with implementing new
functionalities and his supply of motivation and inspiration.\\
\item [\textbf{Evangelos Paradas}] for his guidance trough the architecture of the TS
and pointing me to useful resources.\\
\item [\textbf{Simone Bologna}] for his enthusiasm and patience when finding bugs,
and his steady supply of ideas.
\end{labeling}

Furthermore I would like to express my thanks to the entire Online Software team
for the freedom and trust I've been given that allowed this project to get as
far as it has.



% Can use something like this to put references on a page
% by themselves when using endfloat and the captionsoff option.
\ifCLASSOPTIONcaptionsoff
  \newpage
\fi

\bibliographystyle{IEEEtran}
\bibliography{chapters/bibliografie}

% \begin{IEEEbiography}
%   [{\includegraphics[width=1in,height=1.25in,clip,keepaspectratio]{glenn_dirkx}}]{Glenn~Dirkx}
% was born on July 4, 1992. He received a Bachelor degree in electronics-ICT
% from the Katholieke Hogeschool Kempen (KHKempen), Belgium in 2013.
% \end{IEEEbiography}
%
% \begin{IEEEbiography}
%   [{\includegraphics[width=1in,height=1.25in,clip,keepaspectratio]{christos_lazaridis}}]{Christos~Lazaridis}
% was born on January 17, 1980. He received a M.Sc. degree in Physics from
% the University of Wisconsin-Madison in 2007. In 2011 he received his Ph.D.
% at the University of Wisconsin-Madison in the faculty of Physics.
% He currently works at CERN as a developer for the Level-1 trigger online software
% group for the CMS experiment at CERN.
% \end{IEEEbiography}
%
% % insert where needed to balance the two columns on the last page with
% % biographies
% %\newpage
%
% \begin{IEEEbiography}
%   [{\includegraphics[width=1in,height=1.25in,clip,keepaspectratio]{peter_karsmakers}}]{Peter~Karsmakers}
% was born on April 14, 1979. He
% received a M.Sc. degree in electronics-ICT engineering
% from the Katholieke Hogeschool Kempen
% (KHKempen), Belgium in 2001. In 2004 he received
% the degree of Master in Artificial Intelligence from the
% KULeuven, Belgium. From 2005 to 2010 he was a
% research assistant within the SISTA research group of
% the Department of Electrical Engineering (ESAT), KU
% Leuven, Belgium. In May 2010 he received his Ph.D.
% at the KULeuven in the faculty of Applied Sciences,
% department of Electrical Engineering. He currently is a post-doctoral
% researcher in the AdvISe research group, technology campus Geel,
% KULeuven. His main research interest are machine learning and biomedical
% signal processing. From 2001 he combines his research with teaching in the
% areas of electronics, signal processing and machine learning.
% \end{IEEEbiography}
%
% % You can push biographies down or up by placing
% % a \vfill before or after them. The appropriate
% % use of \vfill depends on what kind of text is
% % on the last page and whether or not the columns
% % are being equalized.
%
% %\vfill
%
% % Can be used to pull up biographies so that the bottom of the last one
% % is flush with the other column.
% %\enlargethispage{-5in}

\end{document}
