\chapter{Development process}
Given the size of the TS and the amount of stakeholders in the proposed changes,
it is important to have a decent planning.

The main objectives are productivity, and making sure developers deliver what is needed most
at a particular moment.

It also would be nice to have a sensible feedback system to allow the stakeholders
to have an influence in the evolution of the TS changes, as they will be the
people who are going to use it.

A modified version of the Scrum method has been used as the development process.
The modifications are focused on providing compatibility with the other development
processes in use by other developers.
Also it has been modified to account for the limited amount of people working on
one task.

\section{Scrum}
Scrum is a relatively new way of developing software projects, although it has
also been used to execute projects not related to development (e.g. construction
projects).

It focuses on delivering functional requirements prioritized on the value it adds
to the project as a whole.

It implements a very strict and repetitive development cycle, usually with a
period of 1 or more weeks. This is called a Sprint. During the beginning of a
Sprint, a set of functional requirements are chosen as a goal, and must be
achieved by the end of the Sprint.

An important distinction to make is that the developers themselves drive this
process. There is no separate person who makes the planning and the set of
requirements for a Sprint on their behalf.
This is where Scrum gets efficient, because the developers after all know best
what is most important and feasible to achieve in one full Sprint.

At the end of a Sprint a set of functional requirements must be met. This means
that particular set of functionality in the project must work in a sense that
the end user can use it. This must be proven by a working demo to the
stakeholders of the project, all of them.

This is also a very important part of Scrum. By demanding the functionality
must work to such an extend that it would be useful for deployment means there
is far less opportunity for hidden errors during actual deployment.
The working demos to all of the stakeholders also provide feedback to
developers at early stages, unlike other systems where stakeholders only get to
see the product all the way at the end of the product development and notice the
developers and stakeholders had some different ideas about functionality.

For more info about Scrum can be found in the book written by one of its inventors,
Jeff Sutherland\cite{scrumbook}.

\subsection{Kanban}
Tasks are divided into distinctive and sequential states. Each task must flow
through each state.
A change in the task results in that task being reset to the initial state.

The following states were chosen for a task:
\begin{labeling}{In productionnnn}
\item [\textbf{Backlog}] This is a list of all the tasks that need to be done.
They are not part of a Sprint, but are the list of candidate tasks for a Sprint.\\
\item [\textbf{To do}] Tasks get moved from the backlog to `To do` when they are
selected to be part of the currently starting Sprint. This list represents the
set of tasks that need to be competed (i.e. be in the `Done` state) by the end of
the Sprint.\\
\item [\textbf{In progress}] When someone is working on a particular task, it is
moved from `To do` to `In progress`.\\
\item [\textbf{In review}] A task get put into `In review` when it is considered
ready for use. At this stage another developer double checks the new functionality.
The main objective is detecting missing features or a misunderstood implementation of it.\\
\item [\textbf{Testing}] At this point the code of a task is pushed to the SVN
repository. The relevant code is then recompiled and tested by a few experts (i.e.
people who will be using this panel).\\
\item [\textbf{Done}] After testing is complete, a task is considered `Done` and
awaits a new software release to be put in `In production`.\\
\item [\textbf{In production}] Once a new release of the TS is pushed to production
systems, the appropriate tasks are moved to `In production`. A task in this state
can be deleted from the point it can be reasonably assumed the relevant features
are stable.\\
\end{labeling}

Trello (\url{https://trello.com/}) has been the tool of choice to implement this
Kanban board (see figure \ref{fig:Kanban}).

\begin{figure}
  \centering
  \includegraphics[width=\textwidth]{images/Kanban}
  \caption{Screenshot of the Trello Kanban board used during development}
  \label{fig:Kanban}
\end{figure}

\subsection{Workflow}
The Scrum process has been modified to account for the tiny number of developers.

Every week a list of functional requirements is made, preferably this does not
encompass any technical goals and thus only contains goals towards end-user
functionality. These are formed into tasks and get put into the backlog.

This list is then sorted according to urgency and importance (urgency takes
precedence over importance).
After sorting, the backlog items are considered to be put into `To do` status up
until a point the Sprint contains enough tasks.

After the tasks have been done, they are in the `In review` stage. Where they will
be either reviewed internally or reviewed during the weekly or monthly meetings
depending on the importance of the feature.

\section{Version Control}
Apache Subversion (SVN) is used to implement version control with the online
software source code.

It is accompanied by a web based ticketing system based on Trac (\url{https://trac.edgewall.org/}).

The repository structure follows all common best practices. The `trunk` folder
contains strictly working and tested code and is used to perform the nightly builds.
It has a `branches` folder containing any pending bug fixes or added features.
Branches follow the `username\_foldername\_ticket\#` naming convention.
The repository also has a `tags` folder, containing working copies of the source
code that are associated with a specific version (e.g. 2.0.1).

The versioning system uses three numbers to signify major, minor, and patch changes,
respectively.

This repository can be found at \url{https://svnweb.cern.ch/trac/cactus}
