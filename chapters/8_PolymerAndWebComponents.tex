\chapter{Polymer}
Polymer allows a developer to declare new web Components. This is used to generate
the custom web interfaces.

This chapter describes how Polymer is used and what developers can do with it.

\section{From C++ to Polymer}
As stated in chapter \ref{Separation of Concerns}, C++ should only be used for
data generation.

However, to keep compatibility with legacy code, the string buffer
system is still used. Therefore the interface must be initiated like so:
(this example is taken from the Flexbox layout example in the Subsystem Supervisor)
\fvset{frame=single}
\begin{pyglist}[language=cpp,numbers=left,numbersep=5pt,fontsize=\small]
#include "subsystem/supervisor/panels/FlexboxLayout.h"
#include "ajax/PolymerElement.h"

using namespace subsystempanels;
FlexboxLayout::FlexboxLayout( tsframework::CellAbstractContext* context,
                              log4cplus::Logger& logger)
:tsframework::CellPanel(context,logger) {
  logger_ = log4cplus::Logger::getInstance(logger.getName() +".FlexboxLayout");
}

void FlexboxLayout::layout(cgicc::Cgicc& cgi) {
  remove();
  add(new ajax::PolymerElement("flexbox-layout"));
}
\end{pyglist}
\fvset{frame=none}

There is still a tiny piece of C++ code that renders the initial HTML
code, at line 13.
It merely specifies the name of the Web Component that renders the panel interface
for the Flexbox layout example panel.

\section{From Polymer to C++}
\label{From Polymer to C++}
The previous example did not contain any data generation.

In order to add support for a callback, it must be registered like so:
(this example is taken from the Form example panel in the Subsystem Supervisor)
\fvset{frame=single}
\begin{pyglist}[language=cpp,numbers=left,numbersep=5pt,fontsize=\small]
  #include "subsystem/supervisor/panels/FormExample.h"
  #include "ajax/PolymerElement.h"
  #include "ajax/toolbox.h"
  #include "json/json.h"

  using namespace subsystempanels;
  FormExample::FormExample( tsframework::CellAbstractContext* context,
                            log4cplus::Logger& logger)
  :tsframework::CellPanel(context,logger) {
    logger_ = log4cplus::Logger::getInstance(logger.getName() +".FormExample");
  }
  FormExample::~FormExample() {
    remove();
  }

  void FormExample::layout(cgicc::Cgicc& cgi) {
    remove();
    setEvent("submit", ajax::Eventable::OnClick, this, &FormExample::submit);
    add(new ajax::PolymerElement("form-example"));
  }

  void FormExample::submit(cgicc::Cgicc& cgi,std::ostream& out) {
    std::map<std::string,std::string> values(
      ajax::toolbox::getSubmittedValues(cgi)
    );

    Json::Value root(Json::arrayValue);

    for(
      std::map<std::string,std::string>::iterator i(values.begin());
      i != values.end(); ++i
    ) {
      Json::Value someresult;
      someresult["name"] = i->first;
      someresult["value"] = i->second;
      root.append(someresult);
    }

    out << root;
  }
\end{pyglist}
\fvset{frame=none}

Now a callback `submit` is registered, which will execute the FormExample::submit
function and return whatever output it generated.

These callbacks can be implemented in Polymer using the `ts-ajax` or the `auto-update`
element in the `common-elements` package.

\fvset{frame=single}
\begin{pyglist}[language=html,numbers=left,numbersep=5pt,fontsize=\small]
<dom-module id="form-example" >
  <template>

    <ts-ajax id="ajax"
             callback="submit"
             data="{{response}}"
             parameters='["sometextinput"]'
             sometextinput={{text1}}></ts-ajax>
    <paper-input label="some text input"
                 value="{{text1}}" ></paper-input>

    <paper-fab icon="send" title="send" on-click="submit" ></paper-fab>

    <h1>server response:</h1>

    <template is="dom-repeat" items="[[response]]" as="item" >
      <p>[[item.name]] = [[item.value]]</p>
    </template>

  </template>
  <script>
    Polymer({
        is: 'form-example',

        properties: {
          response: {
            type: String,
            value: undefined
          }
        },

        submit: function() {
          this.$.ajax.generateRequest();
        }
    });
  </script>
</dom-module>
\end{pyglist}
\fvset{frame=none}

\section{Polymer features}
Polymer is bundled with a lot of useful functionality out of the box.
The most used ones in this project are described here.

A more detailed explanation can be found at the Polymer docs\cite{polymerdocs}
\subsection{Properties}
Polymer elements can contain custom properties defined by the developer.
A property can be an object, a date, a string, a number, an array, a boolean, or a function.
They can be configured to be read-only, to spawn events and/or execute functions
when their value changes.

Properties are useful to display to, or get data from, the user.
\subsection{Data binding}
Data binding is used to connect properties between web components or to bind a
property to a control like an input box or a button value.

It is also useful to automatically fetch data from the server and put it in the
interface without much effort. An example of this can be found in chapter \ref{From Polymer to C++}.
\subsection{Lifecycle callbacks}
A web component can react to lifecycle events. This allows a developer to write
code to be executed when a new instance of the web component is created, or
when it is placed inside the DOM tree, or removed from it.

A useful example is the `auto-update` element, which declares an interval on the
`attached` event, and removes it on the `detached` event.
\subsection{Styling}
AjaXell provides developers with a theme. This is provided in the form of a
web component called `reset-css`. It makes other web components conform to the
AjaXell them (e.g. button size, fonts, colors, \ldots).

Also Polymer gives developers the possibility to use CSS functionality that is
not yet implemented in all browsers, but which are very useful for web components.

One of them is CSS variables and mixins, which allows a developer to define variables in CSS.
Notable is that these variables are inherited just like normal properties.
This is how the AjaXell theme file defines its theme colors.
\fvset{frame=single}
\begin{pyglist}[language=css,numbers=left,numbersep=5pt,fontsize=\small]
:root {
  --primary-color: #00671a;
  --warning-color: yellow;
  --text-color: black;
  --fancy-gradient: {
    background: linear-gradient(to bottom,
                                #1e5799 0%,
                                #2989d8 50%,
                                #207cca 51%,
                                #7db9e8 100%);
  }
}
paper-button {
  color:var(--text-color);
  --paper-button-ink-color:var(--primary-color);
}
paper-button.warning {
  --paper-button-ink-color:var(--warning-color);
}
div[fancy] {
  @apply(--fancy-gradient);
}
\end{pyglist}
\fvset{frame=none}

A Polymer element can include this theme by including it in its template.
\fvset{frame=single}
\begin{pyglist}[language=html,numbers=left,numbersep=5pt,fontsize=\small]
<link rel="import" href="/ts/common-elements/reset-css/reset-css.html" >
<dom-module id="my-element" >
  <template>
    <style include="reset-css" ></style>
    <paper-button>I am themed!</paper-button>
  </template>
  <script>Polymer({is: 'my-element'});</script>
</dom-module>
\end{pyglist}
\fvset{frame=none}

\subsection{Events}
An element can define define events. It can also react to any event spawned by
elements in its template using the following syntax:

\fvset{frame=single}
\begin{pyglist}[language=html,numbers=left,numbersep=5pt,fontsize=\small]
<dom-module id="my-element" >
  <template>
    <!-- paper-button fires 'click' event -->
    <paper-button on-click="clicked" >Click me!</paper-button>
  </template>
  <script>
    Polymer({
      is: 'my-element',
      clicked: function(e, detail) {
        this.fire('kick', {kicked: true});
      }
    });
  </script>
</dom-module>
<my-element></my-element>
<script>
  document.querySelector('my-element').addEventListener('kick', function (e) {
    if (e.detail.kicked) {
      console.log("kick event fired");
    };
  });
</script>
\end{pyglist}
\fvset{frame=none}

\subsection{Behaviors}
Behaviors in Polymer is a way to share JavaScript code among multiple elements.
A useful example of this is the `throws-toast` behavior in common-elements,
which allows an element to show notifications to the user.

The throws-toast.html behavior definition looks like this:
\fvset{frame=single}
\begin{pyglist}[language=html,numbers=left,numbersep=5pt,fontsize=\small]
<script>
throwsToast = {
  /**
   * Throw a message to the central toaster.
   * The `Toast` object accepts the following properties:<br>
   * type (String): Can be 'info', 'warning', or 'error'.<br>
   * message (String | HTMLElement): The actual message to show.<br>
   * options (Array of Strings) (optional): This will force the user
   * to stop what they're doing and choose one of the supplied
   * options. The chosen option is returned.
   *
   * @param {Toast} toast Can be 'info', 'warning', or 'error'.
   * @param {function} callback The function to invoke when one of
   * the options has been clicked.
   * Takes the option string as argument.
   * @return {Void | String}
   */
  throwToast: function(toast) {
    if (typeof toaster === "undefined") {
      console.error(this, "toaster is undefined");
    } else {
      toaster.throwToast(toast, this);
    }
  }
};
</script>
\end{pyglist}
\fvset{frame=none}

An element can implement it like this:
\fvset{frame=single}
\begin{pyglist}[language=html,numbers=left,numbersep=5pt,fontsize=\small]
<link rel="import" href="/ts/ajaxell/html/elements/ag-toaster/throws-toast.html" >
<dom-module id="notifications-demo" >
  <template>
    <paper-button warning on-click="showWarning" >show warning</paper-button>
  </template>
  <script>
    Polymer({
      is: 'notifications-demo',
      behaviors: [throwsToast],
      showWarning: function showWarning() {
        this.throwToast({
            'type': 'warning',
            'message': 'this is a warning at t=' + new Date().getTime(),
            'callback': function callback(response) {
                console.log("callback successfull: ", response);
            }
        });
      },
    });
  </script>
</dom-module>
\end{pyglist}
\fvset{frame=none}
