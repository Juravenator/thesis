\chapter{TS upgrade options}
Now that there is a firm grasp of the current state of the software and its
upgrade requirements is achieved, an assessment of the viability of some upgrade options can be performed.

\section{Server-side interface engine}
Old code must run under the new system. Therefore non-backwards
compatible modifications to the C++ code are not possible.

The only options are to extend the amount of C++ classes to represent new
functionality.
There is no restriction in the sense that the old classes have to remain in use.
Therefore it is possible to change the way the interface is programmed
server-side in function of how extensive the additions to the C++ code are.

\section{Client-side interface library}
At client-side the available options are much more extensive.
The old Dojo library can be kept, and any new front-end library can be
added provided this new library does not interfere with the functionality of the
Dojo library.

Note must be taken however that if any of these new libraries try to dictate the
flow of the web interface, it must allow us to alter it to adjust to the page
flow that was used by the old codebase.

\subsection{Upgrading Dojo}
Currently the used Dojo version is v0.4. It would seem logical to just upgrade
the front-end library to the latest Dojo version, adjust the appropriate C++
classes, and create new ones for the added functionality resulting from the upgrade.
There are however a few problems with this approach.

Firstly, starting from Dojo v0.9, there has been a major Dojo API rewrite.
This would mean an extensive rewrite of the existing C++ classes is needed, which
provides for a very high risk of compatibility problems with legacy panels as
they might anticipate some hacky combination of legacy code that would now have
dissapeared.

Secondly, it would not have solved any of the problems described in chapter \ref{Problems with TS 2.0},
except for the browser compatibility problem.

\subsubsection{Running multiple versions of Dojo}
Running Dojo v0.4 and v0.9 in parallel is also not possible. There is obvious
overlap of code and running these in parallel essentially means entirely
refactoring one of the versions, which is undesirable as it is not maintained
by us and thus not up to us to perform major modifications to.

However, if possible, it would provide easy migration of legacy code as is would
probably mean a panel developer would just have to specify wether to use the new
codebase or not.

\subsubsection{Dojo 2.0}
Currently the latest version of Dojo is v1.10. However the Dojo community is
finalising version 2.0. This release, as the version suggests, brings vast
modifications to the library.

This version is to be released during spring 2016, which is about a year too late
for us to consider. The development started in June 2015 and is expected to be
finished before spring 2016. Currently Dojo 2.0 is not stable enough to be
considered.

If the software would be upgraded to use Dojo 1.10 now, the new TS would be stuck on an outdated
version of Dojo again only months after its release, and no progress will have
been made.
The community would migrate to Dojo v2.0 while the TS stays behind and face all the
current problems all over again.

\subsubsection{Browser compatibility}
Dojo v1.10 claims the following browser compatibility:
\begin{itemize}[noitemsep]
\item Firefox 3.6-20
\item Safari 5-6
\item Chrome 13-26
\item IE 8-10
\item Opera 10.50-12 (Dojo core only)
\end{itemize}
This nicely covers and extends the minimum requirements.

\subsection{jQuery}
One of the initial ideas was to replace Dojo with a library like jQuery or Zepto,
accompanied by a CSS framework like bootstrap, foundation, semantic-ui, susy,
material ui, gumby, Yahoo Pure, or UIKit.

This is a very low-level approach, and for this reason it is actually the safest
to ensure compatibility with Dojo 0.4.

However, this approach does not have much more than that to offer.
It is not a framework, it is a collection of helper classes and functions.
Because it is not a framework, advanced functionality will either be implemented
by heavy code duplication or by an in-house developed library.

This will lengthen the initial development time and will also mean longer
development times for interface developers.

This solution will not leverage the interface developers like a proper
front-end framework will be able to. However, being the most safe option
in terms of compatibility, this presents a safe back-up option if other options
end up being incompatible with current needs.

\subsubsection{Compatibility with Dojo}
Zepto and jQuery functions are fully contained within the `\$` namespace.
This guarantees the absence of any code overlap with Dojo.

The CSS framework however can still present some problems. Both will try to
style common elements like buttons and links, and a decision will have to be
made in how to approach this overlap.
\subsubsection{Browser compatibility}
jQuery combined with the most compatible CSS library (Bootstrap) yields the
following browser support:
\begin{itemize}[noitemsep]
\item Firefox (2 latest versions)
\item Safari 7
\item Chrome (2 latest versions)
\item IE 9
\end{itemize}
Note the `last 2 versions` support for Firefox. This could present problems
depending on the age of the used Firefox browser. However this list enumerates
browsers that have been tested, and since Firefox is generally a `good citizen`
in the world of web browsers it can be assumed this will present no problems. The
fact that IE9 is supported strengthens this assumption.

\subsection{AngularJS}
Angular.js is the most popular front-end web application design framework these
days. It is very powerful and has an extensive developer community.

It is however not very agnostic about how to design a web app. It makes
assumptions, and this could make integrating this with Dojo problematic.


\subsubsection{Learning curve}
Angular.js is very powerful, but this power is accompanied by a rather steep
learning curve\cite{angular_learningcurve} that would make it difficult for people who are not full-time
web application developers to develop panels for the TS.

Among these problems are confusing terminology (e.g. things called `constructors`
that are not constructors), function parameter based dependency injections that
make break minification tools and introduce unnecessary complexity, and an
unclear scoping of variables.

Giving that the developers of panels are not programming experts, the learning
curve must not be too high, it will greatly increase the required development
time to create custom interface panels.

\subsubsection{Running Angular concurrently with Dojo}
Angular 1.x makes the assumption it has complete control over the layout and the
flow of the web application. This will give some issues with the  C++ interface
building logic when building advanced interfaces.
The C++ code is in control of the application. It constructs the page piece by
piece as instructed by the developer and is then to be rendered by the front-end
framework.

\subsubsection{Angular 2.0}
Angular 2 is much more modular, and has a structure quite agnostic and usable to
combine with other frameworks.

Implementing Angular 2.0 seems very viable. Unfortunately it is currently still
in beta status. And while the basics are stable, advanced functionality is still
being debated and developed. This unfortunately limits the use cases for
advanced interfaces for the foreseeable future.

Angular 2 Beta could now be implemented while the final release is awaited, but time
constraints make this an uncomfortable decision to make.

\subsubsection{Browser compatibility}
Angular 2 claims the following browser compatibility:
\begin{itemize}[noitemsep]
\item Firefox (latest development build)
\item Safari 7
\item Chrome (latest development build)
\item IE 9
\end{itemize}
Firefox (latest development build) looks troubling. Further testing shows that
Angular 2 does not work properly in Firefox version <38.

This is very bad, the minimum supported version requirement is v24, 14 versions
lower. Most SLC installations have Firefox 38 installed, so this could be
workable. However, this is the only option that will not pass the minimum
browser support requirements.

\subsection{Web Components}
Web Components\cite{WebComponentsW3C}\cite{WebcomponentsMozilla} are additions to the HTML5 standard. They
enable a developer to develop custom HTML tags, the idea is to mitigate the
`div soup` problem\cite{DivSoup} where the web application's source code increases
exponentially in size as the complexity of the app increases.

This standardizes an approach seen in many modern JavaScript frameworks such as
AngularJS (version 2 in particular), Ember.js, Knockout.js, Dojo, and Backbone.js.
These all allow a developer to declare new `elements` in order to make developing a
smart web application easier.

However, Web Components are a standardized approach to accomplish this. This
means that developers no longer have to worry about major API rewrites like the ones encountered
with Dojo.

Furthermore, a vanilla Web Component is guaranteed to be completely compatible
with any front-end library. A Web Component is in essence and extra HTML tag
and is indistinguishable from a `normal` HTML tag to a front-end framework.

Web Components consist of the following standards:
\begin{labeling}{Custom Elements}
\item [Custom Elements] This standard allows developers to define their own
HTML elements.
\item [HTML Imports] This standard provides a way to import an HTML document,
much like JavaScript and CSS files are currently imported.
\item [Templates] This standard defines `HTML Templates` and allows HTML code to
be reused as needed.
\item [Shadow DOM] This standard provides a way to have multiple independent
HTML DOM trees inside one hierarchy by providing a `shadow root`.
\end{labeling}

\subsubsection{Polymer}
Polymer is a relatively new library, built directly on the Web Components
standards, developed by Google. It represents the way Google thinks Web
Components should be used.

It is very similar to Angular 2 in most respects. For example, they share the
same data binding syntax.

The reason Polymer is very useful is that it has the potential
to allow us to introduce proper Separation of Concerns (SoC) principles
(see chapter \ref{Separation of Concerns}) to the development environment.

\subsubsection{Browser compatibility}
Web Components are a relatively new set of standards and are currently only
supported by Google Chrome.

However, the webcomponents.js project (\url{https://github.com/webcomponents/webcomponentsjs})
aims to polyfill the Web Components standards.

Using this polyfill, browser support can be extended to the following list:
\begin{itemize}[noitemsep]
\item Firefox (latest stable build)
\item Safari 7
\item Chrome (latest stable build)
\item IE 10
\end{itemize}
This list is very similar to Angular 2's compatibility list. However, testing
now concludes that support by Mozilla Firefox goes back all the way to v24, our
minimum requirement.

Note that this means that Polymer has better browser support than Angular 2.
This is curious, as Polymer uses more recent technologies than Angular 2.

\subsubsection{Running Polymer concurrently with Dojo}
In a browser with native Web Components support (i.e. the
webcomponents.js polyfill is not needed), it is guaranteed to have no conflicts between Dojo
and Polymer. This is because Polymer merely adds extras to the Web Components
standards and is all contained in the `Polymer()` JavaScript function.

The webcomponents.js polyfill should also not present conflicts, as most of these
polyfills are transparent.
The polyfill defines the `document.registerElement()` function if it doesn't exist,
manually imports HTML Imports if the browser does not support it natively,
manually stamps `template` elements, and defines the `element.createShadowRoot()`
with an approximation to the Shadow DOM spec, called `Shady DOM`, if it doesn't exist \\
(\url{https://www.polymer-project.org/1.0/articles/shadydom.html}).

Some quick tests with Firefox v24 confirms that these JavaScript libraries do
not present any conflict whatsoever with the Dojo library.

Another possible advantage is that this will probably encounter very little
problems with CSS code. As every common HTML element like buttons and links are
replaced in Polymer with a more powerful Web Component version (e.g. <paper-button>
as a replacement to <input type="button" />).

\subsection{React.js}
React is a JavaScript library, designed by Facebook Inc., that uses a `virtual DOM`
system to abstract away complexity from developers when creating interfaces.

It tries to achieve the same goals as the Web Components standards, however it
does not follow these standards but uses custom technologies. For example it
uses JSX to define templates rather than the HTML Templates standard.

It has the ability to render server-side and client-side.
Server-side rendering is great for web-apps that need good SEO (Search Engine
Optimization), however the TS is an internal app. Also this would mean big
changes in the server-side code need to be made.

React is an open-source project, but it has a questionable license.

\section{Chosen upgrade path}
\subsection{Front-end library}
Things were very close between Angular 2 and Polymer.
They are both the most powerful tools available for front-end interface building
today.

Angular 2 inherits its reputation of robustness, stability, and enterprise-level
code from its predecessor, Angular 1.x.

Polymer has the advantage of being essentially a small `sugaring` layer over
an established W3C standard. This gives us the advantage of robustness against
changes as time advances.
Also the backwards compatibility with older browser versions is more extensive
with Polymer and the webcomponents.js polyfill than with Angular 2.

The agnostic nature of Polymer will also make future updates easier. As
compatibility concerns will be less of an issue.

Seeing that Angular 2 and Polymer try to solve the same problems and share some
code syntax, combined with the fact that Polymer is the more standardized of the
two and Polymer being the more compatible of the two has led to the conclusion
that Polymer is the optimal choice for this project.

React is a notable contender. But the lack of using standards and the custom
license are big drawbacks.

\subsection{Back-end C++ codebase}
The current approach of designing interfaces can be adapted to support the
Separation of Concerns (SoC) principles (see chapter \ref{Separation of Concerns}).

The C++ code will no longer be in charge of defining interface layout, it will
focus and be enhanced for data generation.
The interface will be rendered on client-side using Web Components and its
templates.
Since the main job of XDAQ is Data Acquisition and not interface generation this
change of architecture seems suitable.

Existing C++ code will be kept and parts of it can be enhanced to render a web
component rather than Dojo code when that Dojo code has stopped working, like
the Dojo modal dialog (see figure \ref{fig:ff38} and \ref{fig:ff44} in chapter
\ref{Interface degradation}).
